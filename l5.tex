% chktex-file 44
% chktex-file 18
% chktex-file 8
\documentclass{article}
\usepackage{amssymb}
\usepackage{amsfonts}
\usepackage{amsmath}
\usepackage{cancel}
\usepackage{graphicx}
\graphicspath{ {./assets/} }
\author{Enlai Li}
\title{MATH240 -- Lecture 5}
\date{January 18, 2023}
\begin{document}
\maketitle
\section{Predicate Logic}
\begin{center}
    p = "All men are mortal"
\end{center}
is an atomic proposition from the point of view of propositional logic (cannot be decomposed into simpler propositions).
It still has some structure:
\begin{itemize}
    \item "men" and "mortal" are predicates
    \item "All" is a quantifier
    \item "are" is a copula
\end{itemize}

A predicate is something that is true, or false, about a subject (which may vary)
\\ex: "mortal"
\begin{itemize}
    \item Socratos is mortal: True
    \item Zeus is mortal: False
\end{itemize}

Mathematically, a predicate is a function
\begin{gather*}
    P: x\in U \rightarrow Bool:\{T,F\} \quad U: \text{Universe of discourse}\\
    \text{Subject } \rightarrow \text{ Truth value}
\end{gather*}

In predicate logic, instead of dealing with propositional variables, we deal with predicates, P(x) and our variables x range over any universe U
\begin{gather*}
    P(x): \text{"} x \ge 0 \text{"} \rightarrow \text{ When you give a value to x, you find a proposition } \\
    P(2): \text{"} 2 \ge 0 \text{"}: True \\
    P(-1): \text{"} -1 \ge 0 \text{"}: False \\
\end{gather*}

Quantifiers:
\begin{itemize}
    \item $\forall$: "for all": Universal quantifier
    \item $\exists$: "there exists": Universal quantifier
\end{itemize}

We can bind a variable (in a proposition) to a quantifier, so that it can no longer be freely set

\end{document}