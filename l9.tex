\documentclass{article}
\usepackage{amsmath}
\usepackage{amsfonts}
\author{Enlai Li}
\title{MATH 240 \textemdash \ Lecture x}
\date{February 2, 2023}
\begin{document}
\maketitle

\section{Bijection Principle}
If A, B are finite sets: \\

\[|A| = |B| \iff there exists a bijective function f: A \rightarrow B\]

Ex: Show that for a finite set A
\[|P(A)| = 2^{|A|}\]

\[B_n = \{0, 1\} x \{0,1\} - x\{0, 1\} \\
    = \{0,1\}^n\]

So we just need to find a bijection
\[
    f: P(A) \rightarrow B_n\]
in order to conclude (with the bijection Principle) that:
\[|P(A)| = 2^n\]
Let $x \in P(A)$, then $x \subseteq A$ \\
Let $A = \{a_1, a_2, \ldots, a_n\}$
Define: $f(x) = (b_1, b_2, \ldots, b_n)$

\begin{equation}
    \text{where $b_i$} =
    \begin{cases}
        1 & \text{if $a_i \in X$}    \\
        0 & \text{if $a_i \notin X$}
    \end{cases}
\end{equation}
Ex: \[A=\{1,2,3,4\}, X=\{1,4\} \Rightarrow f(x) = (1,0,0,1)\]

This f is \underline{clearly} invertible
\[f^{-1}(b_1, b_2, \ldots, b_n) = \{a_i \in A | b_i = 1\}\]
invertible $\Rightarrow$ bijective
\[f^{-1}(f(X)) = X \\
    f(f^{-1}(b)) = b\]

\section{Infinite Cardinalities}
Let's extend the bijection principle to Infinite sets
Def: Two sets A and B have the \underline{same cardinality} if there is a bijection $f:A\leftarrow B$

We then write \[|A| = |B|\]
Ex:
\[ \mathbb{N} = \{0,1,2,3,4,\ldots \}\]
\[ \mathbb{E} = \{0,2,4,6,8,\ldots \}\]


Theorem: There is \underline{no} bijection between $\mathbb{N}$ and $\mathbb{R}$ for $(0, 1)$

That $\mathbb{R}$ is strictly larger than $\mathbb{R}$
\[\mathbb{N} < \mathbb{R}\]
There are at least two different infinities! (In fact, there are infinitely many)

Proof: (Canton's Diagonal Argument)

By contradiction. Assume there is a bijection
\[f:\mathbb{N} \leftarrow (0,1)\]

Note: any $x \in (0,1)$ can be written in decimal notation:
\[x = 0, a_1, a_2, a_3, \ldots (a_i \in 0, 1, \ldots, 9)\]

\[Ex: \frac{1}{3} = 0.33333\ldots \\
    a_i = 3, \forall_i\]

\begin{align*}
    f(0) = 0, a_0
\end{align*}

Now consiter the following nunber: \[C = 0.C_0C_1C_2C_3\ldots\]
where
\begin{equation}
    \text{where } c_i =
    \begin{cases}
        4 & \text{if} a_{ii} \neq 4 \\
        2 & \text{if} a_{ii} = 4
    \end{cases}
\end{equation}

What matters is $c_i \neq a_ii (\forall_i) \Rightarrow C \neq f(n), \forall n \in \mathbb{N}$

That's a contradiction!
$c \in (0,1)$ but it is not in the "list" (which should have been complete) $\square$

\subsection{Some remarks}
\begin{enumerate}
    \item Sets in bijection with $\mathbb{N}$ are called countable sets, ex:
          \[\mathbb{N}, \mathbb{E}, \mathbb{O}, \mathbb{Z}, \mathbb{Q}\]
          They are the "smallest" kind of infinite sets
    \item Sets in bijection with $\mathbb{R}$ are called "continuous" \\
          ex: $\mathbb{R}, (0,1)$, any real interval, \ldots
    \item There are sets that are larger than $\mathbb{R}$ ex: \\
          \[\mathbb{R} \subset |P(\mathbb{R})| \subset |P(P(\mathbb{R})) \subset \ldots\]
          The proof of that is analogous to the one we just did
    \item Is there a set $X$ with $|\mathbb{N}| < |\mathbb{X}| < |\mathbb{R}|$? \\
          This problem is called the continuum hypothesis and is known to be undecidable!
\end{enumerate}

\subsection{Relations}
Ex: Unit circle
\[C = \{(x,y) \in \mathbb{R}^2 | x^2 + y^2 = 1\}\]

Def: A \underline{relation} on a set $A$ is a subset $R \subseteq A \times A$

Ex: Function $f: A \rightarrow A$
\[\text{Graph } (f) = \{(x,y) \in A \times A | y = f(x)\}\]
That is a relation which satisfies the vertical line test \ldots

Ex: Equality on a set A:
\[A = \{(x, x) | x\in A\}\]
Ex:
\begin{align}
    (5,5) \in =_\mathbb{Z} : 5 =_\mathbb{Z} 5 \\
    (2,5) \in \neq_\mathbb{Z} : 2 \neq_\mathbb{Z} 5
\end{align}

\subsection{Infix Notation}
\begin{align}
    s
\end{align}

Ex4: $U$: some universe of sets \ldots
\[A \sim B \Rightarrow |A| = |B| \text{ meaning } \exists f: A \rightarrow B \text{ bijective}\]
$\sim$ is a relation on $U$:
\[\mathbb{N} \sim \mathbb{Z}\]

\begin{align}
    3
\end{align}


\end{document}