% chktex-file 44
% chktex-file 18
\documentclass{article}
\usepackage{amssymb}
\usepackage{amsfonts}
\usepackage{amsmath}
\usepackage{cancel}
\usepackage{graphicx}
\graphicspath{ {./assets/} }
\author{Enlai Li}
\title{MATH240 \textendash{} Lecture 3}
\date{January 11, 2023}
\begin{document}
\maketitle
\section{Mathematical logic}
\begin{itemize}
    \item [Premise $\rightarrow $] All men are $\underbrace{mortal}_{M}$
    \item [$\rightarrow $] $\underbrace{Socrates}_S$ is a $\underbrace{man}_{H (human)}$
    \item [Conclusion $\rightarrow $] Therefore Socrates is mortal
\end{itemize}
All x such that x is H is M.\\ S is H\\ Therefore S is M
\begin{table}[h!]
    \begin{center}
        \begin{tabular}{l}
            $\forall x, H(x) \Rightarrow M(x)$ \\
            $H(S)$                             \\
            \hline{}
            $M(S)$                             \\
        \end{tabular}
    \end{center}
\end{table}

\section{Propositional logic (True/False)}
\subsection{Atomic propositions}
The building block of propositional logic are propositions: statements that are either true or false. \\
ex:
\begin{itemize}
    \item [p:] 21 is a multiple of 7 (True proposition)
    \item [q:] $2+2=5$ (False proposition)
    \item [r:] there exists an extraterrestrial life form (proposition)
\end{itemize}
\subsection{Compound propositions}
Atomic propositions combined with logical connectors \\ ex:
\begin{gather*}
    2+2=5 \text{ and } \text{"there exists and extraterrestial lifeform}\\
    = q \land r
\end{gather*}

\subsection{Logical connectors}
\subsubsection{conjunction ($\land$) and}
\begin{table}[h!]
    \begin{center}
        \begin{tabular}{c|c|c}
            p & q & $p \land q$ \\
            \hline
            T & T & T           \\
            T & F & F           \\
            F & T & F           \\
            F & F & F
        \end{tabular}
    \end{center}
\end{table}
\subsubsection{Negation ($\lnot$) not}
\begin{align*}
    \lnot p & = \text{ Not } p = \overline{p} \\
    q       & = "2+2=5"                       \\
    \lnot q & = "2+2 \neq 5"
\end{align*}
\begin{table}[h!]
    \begin{center}
        \begin{tabular}{c|c}
            p & $\lnot p$ \\
            \hline
            T & F         \\
            F & T
        \end{tabular}
    \end{center}
\end{table}
\subsubsection{Disjunction ($\lor$) or}
\begin{table}[h!]
    \begin{center}
        \begin{tabular}{c|c|c}
            p & q & $p\lor q$ \\
            \hline
            T & T & T         \\
            T & F & T         \\
            F & T & T         \\
            F & F & F
        \end{tabular}
    \end{center}
\end{table}

\subsubsection{More complex propositions}
\[
    \lnot (\lnot p \land \lnot q) = ?
\]
\begin{table}[h!]
    \begin{center}
        \begin{tabular}{c|c|c|c|c|c}
            p & q & $\lnot p$ & $\lnot q$ & $\lnot p \land \lnot q$ & $\lnot (\lnot p \land \lnot q)$ \\
            \hline
            T & T & F         & F         & F                       & T                               \\
            T & F & F         & T         & F                       & T                               \\
            F & T & T         & F         & F                       & T                               \\
            F & F & T         & T         & T                       & F
        \end{tabular}
    \end{center}
\end{table}

\subsubsection{Set laws and logical equivalence}
Set laws translate into logical equivalence:
\begin{table}[h!]
    \begin{center}
        \begin{tabular}{c|c}
            Set            & Logic     \\
            \hline
            $U$            & True      \\
            $\emptyset $   & False     \\
            $\cup$         & $\land$   \\
            $\cap$         & $\lor$    \\
            $\overline{A}$ & $\lnot A$ \\
        \end{tabular}
    \end{center}
\end{table}

Since $p \lor q$ has the same truth table as $\lnot (\lnot p \land \lnot q)$, they are logically equivalent:
\[
    p \lor q \equiv \lnot (\lnot p \land \lnot q)
\]

\subsubsection{Exclusive or ($\oplus $) xor}
Def: $p \oplus q \equiv (p \lor q) \land \lnot (p \land q)$

\begin{table}[h!]
    \begin{center}
        \begin{tabular}{c|c|c}
            p & q & $p \oplus q$ \\
            \hline
            T & T & F            \\
            T & F & T            \\
            F & T & T            \\
            F & F & F
        \end{tabular}
    \end{center}
\end{table}

\subsubsection{Conditional ($\Rightarrow $)}
Def: $p \Rightarrow q \equiv \lnot (p \land \lnot q) \equiv\lnot p \lor q$\\
$p \Rightarrow q$: "p implies q": "if p then q"\\
ex:
\begin{itemize}
    \item [p:] It rains
    \item [q:] It's cloudy
    \item [$p \Rightarrow q$:] If it rains outside, then it is cloudy
\end{itemize}

\marginpar[]{\raggedright{} $p \Rightarrow q$ is always true when p is false}
\begin{table}[h!]
    \begin{center}
        \begin{tabular}{c|c|c}
            p & q & $p \Rightarrow q$ \\
            \hline
            T & T & T                 \\
            T & F & F                 \\
            F & T & T                 \\
            F & F & T
        \end{tabular}
    \end{center}
\end{table}
\end{document}

