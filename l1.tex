\documentclass{article}
\usepackage{amssymb}
\usepackage{amsfonts}
\usepackage{amsmath}
\author{Enlai Li}
\title{MATH240 -- Lecture 1}
\date{January 3, 2023}
\begin{document}
\maketitle
\section{Set Theory}
$A = \{0, 1, 5\} \text{ finite set with 3 elements}$

\subsection{$\in$ (is element of)}
Notation: $x \in A \rightarrow x \text{ is element of } A$
\begin{gather*}
    A = \{0,1,2,3\} \\
    0 \in A \text{\indent } 42 \notin A
\end{gather*}

\subsection{Set by extension}
\begin{align*}
    \emptyset  & = \{\}                                                                  \\
    \mathbb{N} & = \{0,1,2,\ldots\}                                                      \\
    \mathbb{Z} & = \{\ldots,-2,-1,0,1,2,\ldots\}                                         \\
    \mathbb{Q} & = \text{ Rational numbers (fractions) } = \{0, \frac{1}{2}, \frac{2}{3}
    , 5, \frac{-42}{11}, \ldots\}
\end{align*}

\subsection{Set by comprehension}
Notation: $A = \{x \in U \ | \ P(x) \text{ is true}\}$
\begin{align*}
    \text{even numbers: } E              & = \{\ldots, -4, -2,0,2,4,\dots \}                                                                                      \\
                                         & = \{x \in \mathbb{Z} \ | \ x = 2n \text{ for some } n \in \mathbb{Z} \}                                                \\
    \text{odd numbers: } O               & = \{\dots,-3,-1,3,\dots \}                                                                                             \\
                                         & = \{x \in \mathbb{Z} \ | \ x = 2n+1 \text{ for some } n \in \mathbb{Z} \}                                              \\
    \text{also: }                        & =  \{n+1 \ | \ n \in E \}                                                                                              \\
                                         & = \{x \in \mathbb{Z} \ | \ x = n+1 \text{ for some } n \in E \}                                                        \\
    \text{rational numbers: } \mathbb{Q} & = \{\frac{a}{b} \ | \ a \in \mathbb{Z}, b \in                             \mathbb{Z}, b \neq 0, b > 0, GCD(a,b) = 1 \}
\end{align*}

\subsection{Subsets}
If every element of set $A$is also element of set $B$, then $A$ is subset $B$ \\
Notation: $A \subseteq B$
\begin{gather*}
    A = \{0,1,5\} \subseteq \mathbb{N} \\
    \mathbb{N} \subseteq \mathbb{Z} \subseteq \mathbb{Q}
\end{gather*}
2 sets are equal if $A \subseteq B \text{ and } B \subseteq A$: $A = B$

\subsection{Cardinality}
Number of elements in a set \\
Notation: $\lvert A \rvert$
\[
    A = \{1,3,5\} \text{\indent } \lvert A \rvert = 3
\]
\end{document}