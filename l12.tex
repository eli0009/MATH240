% chktex-file 44
% chktex-file 18
% chktex-file 8
\documentclass{article}
\usepackage{amssymb}
\usepackage{amsfonts}
\usepackage{amsmath}
\usepackage{cancel}
\usepackage{graphicx}
\graphicspath{ {./assets/} }
\author{Enlai Li}
\title{MATH240 -- Lecture 12}
\date{February 10, 2023}
\begin{document}
\maketitle
\section{Last lecture}
\subsection{Euclid's algorithm}
\[
    a,b \in \mathbb{Z} \rightarrow gcd(a,b) = d
\]
\subsection{Bezout's theorem}
\[
    d = gcd(a,b) \rightarrow d = sa + tb
\]
\subsection{Corollary}
If $c|a$ and $c|b$, then $c|gcd(a,b)$

\subsection{Proof}
\begin{gather*}
    c|a \text{ and } c|b \\
    \Rightarrow a = kc \text{ and } b = nc
\end{gather*}
Then
\begin{align*}
    d
     & = gcd(a,b)                                   \\
     & = sa + tb \text{, where } s,t \in \mathbb{Z} \\
     & = sac + tbc                                  \\
     & = c(sk+tn) \Rightarrow c|d
\end{align*}

\section{Coprime}
two integers a,b are coprime if gcd(a,b) = 1
ex:
\begin{itemize}
    \item 42 and 515 are coprime
    \item a = 7, n = not a multiple of 7 \\ $\Rightarrow gcd(7,9)=1$
\end{itemize}
\marginpar[]{\raggedright{} This works for any other prime number}
That's because the only divisions of 7 are 1 (only possibility left) and 7 (not a divisor of n)

\textbf{Theorem} a and b are coprime $\iff 1 = sa+tb$ \\
\textbf{proof} Bezout when d=1
\begin{gather*}
    1=sa+tb \\
    \text{Let } d = gcd(a,b) \\
    \text{ Goal: prove } d = 1 \\
    d|a \text{ and } d|b\\
    \Rightarrow d|sa + tb \text{ (elementary property of 1)}\\
    \Rightarrow d|1\\
    \Rightarrow d = 1
\end{gather*}

\section{prime numbers}
p is prime $\iff p > 1$ and its only positive divisors are 1 and p

ex: \[
    2,3,5,7,11,13,\dots
\]
A number that is not a prime is called composite
\marginpar[]{\raggedright{} n is composite $\iff n = ab \text{, where } a,b>1$}
ex: \[
    42 = 6 \times 7
\]
Prime numbers are interesting in number theory because they are easy to understand, yet they easily lead to very difficult problems

\subsection{Goldbach's conjecture (open since 1742)}
Every even number $n>2$ is the sum of two primes\\
ex:
\begin{gather*}
    42 = 19 + 23\\
    20 = 13+7
\end{gather*}
It's been tested by computers to work up to very large numbers (400 trillions),
no one has proved it

\subsection{Fundamental Theorem of Arithmetic (FTA)}
primes are a fundamental role in number theory as the building blocks of all integers
\\ex: we can write 42 as product of primes
\begin{align*}
    42 & =6 \times 7 \\&= 2 \times 3 \times 7
\end{align*}
We can always decompose a number as a product of primes, in a unique way. We need the following lemma to prove this: \\
\textbf{Lemma} if p is prime and $p|ab$ then $p|a$ or $p|b$\\
\marginpar[]{\raggedright{} We really need p to be prime for this to work}
ex:
\begin{gather*}
    3 | 42 = 6 \times 7 \text{ and indeed } 3 | 6 \\
    \text{Counter-example } 14 | 42 = 6 \times 7 \text{ but } 14 \cancel{\lvert} 6 \text{ and } 14 \cancel{\lvert} 7
\end{gather*}
\textbf{Proof}
\begin{gather*}
    \text{Assume } p | ab, p \text{ is prime } \Rightarrow ab=px \\
    \text{Goal: } p|a \text{ or } p|b \\
    \text{Assume } p \cancel{|} a, \text{ New goal} p | b \\
    \text{Since } p \cancel{|} a \text{ then p and a are coprime}
\end{gather*}
\begin{align*}
    \Rightarrow  1 & = sp+ta         \\
    b              & = spb + tab     \\
                   & = spb+tpx       \\
                   & = p(sb+tx)      \\
                   & \Rightarrow p|b
\end{align*}
\hspace*{\fill} $\square$

Let $n \ge 2$ be an integer, the we can find prime numbers \[
    p_1 \le p_2 \le p_3 \dots \le p_k
\]

such that $
    n =    p_1 \le p_2 \le p_3 \dots \le p_k
$
moreover this list of prime is unique
\textbf{Proof} We must prove existence and uniqueness of the prime factorization of n. We do both in a single proof by strong induction!

\textbf{Base case:} n = 2
\begin{itemize}
    \item Existence: n = 2 (prime)
    \item Uniqueness: $2 =
              p_1  p_2  p_3 \dots  p_k \text{, where }
              p_1 = 2 \text{ and }  p_2  p_3 \dots  p_k = 1$
\end{itemize}
\textbf{Induction step}
Asuume the FTA true for all integers $< b$. We want to prove it for n.
2 cases:
\\
\textbf{ n is prime}: same as base case (replace 2 by n)\\
\textbf{ n is composite}:
\begin{itemize}
    \item Existence: $n = ab, n > a,b \ge \mathbb{Z} $, by induction hypothesis we can write
          \begin{gather*}
              a =p_1  p_2  p_3 \dots  \le p_k \\
              b = q_1  q_2  q_3 \dots \le q_l \\
              \Rightarrow  n = p_1p_2 \dots p_kq_1q_2 q_l
          \end{gather*}
          This is a product of primes! rearrange them in  increasing order and we have a solution
    \item Uniqueness: Assume the two prime decompositions of n
          \begin{gather*}
              n =p_1  p_2  p_3 \dots p_k \text{, where } p_1 \le \dots  \le p_k \\
              n =q_1  q_2  q_3 \dots q_l \text{, where } q_1 \le \dots  \le q_l \\
              p_1 | n \Rightarrow p_1|q_1q_2 \dots q_l \\
              \text{By the lemma}\\
              p_1|q_1 \text{ or } p_1|q_2 \text{ or } \dots \text{ or } p_1|q_l \\
              \Rightarrow p_1=q_1 \text{ or } p_1=q_2 \text{ or } \dots \text{ or } p_1=q_l \\
              \Rightarrow p_1=q_1 \text{, for some } i
          \end{gather*}
          Now we consider the number $\frac{n}{p_1} < n$ by the induction hypothesis, all primes $p_2  p_3 \dots  p_k$ are the same as the primes $q_1  q_2  q_3 \dots  q_l$. All primes $p_1 p_2  p_3 \dots  p_k$ are the same as $q_1  q_2  q_3 \dots  q_l$
          \[
              k=l \text{ and } p_1 = q _{1} \text{ and } \dots \text{ and } p_k = p_l
          \]
          \\
          \hspace*{\fill} $\square$
\end{itemize}
We can regroup repeated factors and write the prime decomposition with exponents (canonical form)
\[
    n = p_1 ^{\alpha_1} p_2  ^{\alpha_2} p_3 ^{\alpha_3} \dots  p_k^{\alpha_k} \text{, where } p_1 < \dots < p_k \text{ and } \alpha_1 > \dots > \alpha_k > 0
\]
ex:
\begin{align*}
    72
     & = 2 \times 36                           \\
     & = 2 \times 2 \times 2 \times 3 \times 3 \\
     & = 2 ^{3} \times 3 ^{2}
\end{align*}
We could in fact allow 0 in the exponents but we could lose uniqueness of the list of primes\\
\textbf{Lemma}
\begin{gather*}
    \text{With all exponents $\le 0$, let }\\
    a = p_1 ^{\alpha_1} p_2  ^{\alpha_2} p_3 ^{\alpha_3} \dots  p_k^{\alpha_k} \\
    b = p_1 ^{\beta_1} p_2  ^{\beta_2} p_3 ^{\beta_3} \dots  p_k^{\beta_k}
    \\ \text{Then}\\
    a | b \iff \alpha_i \le \beta_i \text{, for all } i
\end{gather*}
Ex:
\begin{gather*}
    72 = 3 ^{2} 2^3 \\
    36 = 3 ^{2} 2 ^{2}
\end{gather*}
\textbf{Proof} Suppose $a|b$ then $b=ac$ \\
Let $c = p_1 ^{\alpha_1} p_2  ^{\alpha_2} p_3 ^{\alpha_3} \dots  p_k^{\alpha_k}$, Then $c = ac$
\begin{gather*}
    p_1 ^{\beta_1} p_2  ^{\beta_2} p_3 ^{\beta_3} \dots  p_k^{\beta_k} =p_1 ^{\alpha_1} p_2  ^{\alpha_2} p_3 ^{\alpha_3} \dots  p_k^{\alpha_k} \times  p_1 ^{q_1} p_2  ^{q_2} p_3 ^{q_3} \dots  p_k^{q_k}
\end{gather*}
Exponents are unique (by FTA)
\begin{gather*}
    \beta_1 = \alpha + q_1 \ge \alpha_1\\
    \beta_2 = \alpha + q_2 \ge \alpha_2\\
    \beta_k = \alpha + q_k \ge \alpha_k\\
\end{gather*}
Assume $\alpha _i \le \beta _i, \forall i$
Let \[
    p_i = \beta-\alpha_i, \forall i \ge 0
\]
Let $c = p_1 ^{\beta_1} p_2  ^{\beta_2} p_3 ^{\beta_3} \dots  p_k$
Then $b = ac Ra a | b$
\end{document}
